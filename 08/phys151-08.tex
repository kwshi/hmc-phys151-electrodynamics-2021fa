\documentclass{phys151}

\title{Homework 8}
\date{2021 November 15 (Monday)}
\author{Kye Shi}

\begin{document}

\begin{problem}
  Derive a relationship with the magnetic moment \(\vec m\) and the angular
  momentum of a charged particle moving in a circle.
\end{problem}

\begin{solution}

\end{solution}

\begin{problem}
  Consider two parallel magnetic dipoles \(\vec m\) as shown below.
  \begin{center}
    \begin{tikzpicture}
      \tikzset{
        vec/.style={very thick, arrows=-Stealth},
      }

      \draw[opacity=1/2] (0,0) coordinate(a) -- (30:1.5in) coordinate(b)
      node[midway, below] {\(R\)};
      \draw[vec] (a) -- +(0,3em) node[left, midway]{\(\vec m\)};
      \draw[vec] (b) -- +(0,3em) coordinate (b') node[right, midway]{\(\vec m\)};

      \draw pic[draw, opacity=1/2, pic text=\(\theta\), angle eccentricity=3/2]{angle=b'--b--a};
    \end{tikzpicture}
  \end{center}
  Find the force and torque on either due to the other.
\end{problem}

\begin{solution}

\end{solution}

\begin{problem}
  Prove that, in a process of hysteresis where we vary \(-\vec B_0 \to +\vec
  B_0 \to -\vec B_0\), the energy loss is proportional to the area of the
  hysteresis loop on an \(\vec H\)--\(\vec B\) diagram.
\end{problem}

\begin{solution}

\end{solution}

\begin{problem}
  Consider a sphere of radius \(R\) with uniform \emph{permanent} magnetization
  \(M_0\).
  \begin{subproblems}
    \item For \(r>R\), using \(\grad\times\vec H=0\), introduce a ``potential''
      \(\phi_M\) by \(\vec H=-\grad\phi_M\).  Show that
      \[
        \lapl \phi_M = 4\pi \grad\cdot\vec M \equiv -4\pi\rho_m.
      \]
    \item Show that
      \[
        \phi_M
        = -\int \frac{\grad' \cdot \vec M(\vec x')} {\abs{\vec x-\vec x'}} \d V'
        + \int \frac {\vec M(\vec x') \cdot \hat{\vec x}'} {\abs{\vec x-\vec x'}} \d A'.
      \]
    \item Find \(\vec H\) everywhere.
  \end{subproblems}
\end{problem}

\begin{solution}

\end{solution}

\begin{problem}
  Consider the previous problem but now the magnetization is induced by an
  external uniform field \(\vec B_0\).  The permeability of the sphere is
  \(\mu\).  Find the magnetization \(\vec M\) of the sphere.
  \begin{hint}
    You need very little extra effort given the previous problem's solution.
  \end{hint}
\end{problem}

\begin{solution}

\end{solution}

\end{document}
