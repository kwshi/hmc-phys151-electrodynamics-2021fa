\documentclass{phys151}

\title{Homework 9}
\date{2021 November 21 (Monday)}
\author{}

\begin{document}

\begin{problem}
  Consider a dipole consisting of a point electric charge \(q\) and a point
  magnetic charge \(g\), separated by a distance \(d\).
  \begin{subproblems}
    \item Find the \(\vec E\) and \(\vec B\) fields.
    \item Compute the total angular momentum in \(\vec E\) and \(\vec B\) by
      evaluating the integral of the Poynting vector
      \[
        \vec L = \int \vec r \times \frac {\vec S} {c^2} \d V.
      \]
    \item Derive Dirac's quantization of charge by setting \(\abs{\vec L} =
      \hbar n\) with \(n\in\mathbb Z\).
  \end{subproblems}
\end{problem}

\begin{solution}

\end{solution}

\begin{problem}
  Show that \({{T_{EM}}^\mu}_\mu = 0\).
\end{problem}

\begin{solution}

\end{solution}

\begin{problem}
  Consider a conducting wire carrying a current \(I\) and obeying Ohm's law
  \(\vec J = \sigma \vec E\).  Assume the wire is cylindrical of radius \(R\)
  and length \(L\).  Show that the energy lost in the wire is balanced by the
  flow of EM energy into it.
\end{problem}

\begin{solution}

\end{solution}

\begin{problem}
  Derive \(\grad\cdot\vec j = -\pderiv \rho t\) from Maxwell's equations.
\end{problem}

\begin{solution}

\end{solution}

\begin{problem}
  A solenoid of radius \(R\) with n turns per unit length carries a stationary
  current \(I\).  Two hollow cylinders of length \(l\) are fixed coaxially and
  freely rotating.  One cylinder of radius \(a\) is inside the coil (\(a<R\))
  and carries uniformly distributed charge \(Q\).  The outer cylinder of radius
  \(b\) (\(b>R\)) carries charge \(-Q\).  The current is suddenly switched off
  and the cylinders start to rotate.  Find the angular momenta of the two
  cylinders.
\end{problem}

\begin{solution}
  
\end{solution}

\end{document}
