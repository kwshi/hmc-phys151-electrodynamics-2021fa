\documentclass{phys151}

\title{Homework 5}
\date{2021 October 21 (Thursday)}
\author{}

\begin{document}

\begin{problem}
  Consider the complex potential \(f(z)\) where \(f(z) = u(x) + \i v(z)\) with
  \(u\) being the electric potential \(\phi\) in 2D.  Define \(f(z)\)
  implicitly through
  \[
    x = u + e^{-v} \sin u, \qquad y = v - e^{-v} \cos u.
  \]
  \begin{subproblems}
  \item Show that this function corresponds to a potential between two
    semi-infinite parallel plates with potential \(\pm\pi\), with geometry
    shown in the figure below.  For this, show that
    \begin{subproblems}
    \item \(\phi = \pm\pi\) at \(x = \pm\pi\) with \(y\ge1\);
    \item The equipotentials far away from the edges are as expected.
    \end{subproblems}
  \item Show that the equipotential lines bend away as \(\frac x y = -\tan
    u\) for \(v \to -\infty\); and for \(u = \frac \pi 2\), we have \(x =
    \frac \pi 2 + e^{-y}\) at constant potential.  Sketch the equipotentials
    qualitatively.
  \end{subproblems}
\end{problem}

\begin{solution}

\end{solution}

\begin{problem}
  Sketch the equipotentials for the following complex potentials:
  \begin{subproblems}
  \item \(\Omega(z) = (\alpha-\i\beta) z\) (uniform field)
  \item \(\Omega(z) = \alpha \ln(z-c)\) (point @ \(z=c\))
  \item \(\Omega(z) = \i\beta \ln(z-c)\) (vortex @ \(z=c\))
  \item \(\Omega(z) = (\alpha+\i\beta) \ln(z-c)\)
  \end{subproblems}
  (\(\alpha, \beta\) are real, \(c\) is complex.)
\end{problem}

\begin{solution}

\end{solution}

\begin{problem}
  Consider the two circles
  \begin{align*}
    \Gamma_1 &\colon \abs z = 1, \\
    \Gamma_2 &\colon \abs{z-1} = \frac 5 2.
  \end{align*}
  We want to solve \(\lapl\phi = 0\) in the region between the two circles with
  \(\phi=a\) on \(\Gamma_1\) and \(\phi=b\) on \(\Gamma_2\), where \(a\) and
  \(b\) are constants.
  \begin{subproblems}
  \item Applying the map \(\zeta = \frac {z+1/4} {z+4}\), show that
    \(\Gamma_1\) and \(\Gamma_2\) are mapped to concentric circles.
  \item Show that the complex potential is
    \[
      \chi(\zeta) = 2b-a + \frac {b-a} {\ln 2} \ln \zeta.
    \]
  \item Find \(\phi\).
  \end{subproblems}
\end{problem}

\begin{solution}

\end{solution}

\begin{problem}
  Consider the region bounded by
  \begin{align*}
    \Gamma_1 &\colon \abs z = 1, & \Im z &\ge 0, \\
    \Gamma_2 &\colon \abs z = 1, & \Im z &\le 0.
  \end{align*}
  Solve \(\lapl\phi=0\) with \(\phi=a\) on \(\Gamma_1\) and \(\phi=a+k\) on
  \(\Gamma_2\).  Do this by using the conformal map
  \[
    w = \frac {z-1}{z+1}
  \]
  and the transformation
  \[
    \zeta = \ln w - \i \frac \pi 2.
  \]
  The solution in the \(\zeta\) plane should be very easy.
\end{problem}

\begin{solution}

\end{solution}

\end{document}
