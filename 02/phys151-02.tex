\documentclass{phys151}

\title{Homework 2}
\date{2021 September 27 (Monday)}
\author{}

\begin{document}

\begin{problem}
  \leavevmode
  \begin{subproblems}
  \item Find the Lorentz transformations of charge density \(\rho\) and current
    density \(\vec j\).  (Use \(\mathcal O'\) moving with \(v\) along the
    \(x\)-axis.)
  \item Show that, under Lorentz transformations, total charge
    \[
      Q \equiv \int j^0(0, \vec x) \d V
      = \int j^0(x) \delta(x^0) \d^4 x
    \]
    is invariant; that is \(Q = Q'\) where \(Q' = \int j^{0'}(x')
    \delta(x^{0'}) \d^4 x\).
  \end{subproblems}
  \begin{hint}
    Write  \(j^0(x) \delta(x^0) = j^\mu(x) \pd_\mu \theta(x^0)\), where
    \(\theta\) is the Heaviside (step) function.

    You will also need \({\Lambda^0}_0 \ge 1\).  Show that \(Q'-Q = 0\): first
    show that  \(Q' = \int j^\mu \pd_\mu \theta(x^{0'}) \d^4 x\); then, look at
    \(Q'-Q\) and show that it vanishes if \({\Lambda^0}_0 \ge 1\).
  \end{hint}
\end{problem}

\begin{solution}
  
\end{solution}

\begin{problem}
  Define four-acceleration as \(a^\mu = \deriv {u^\mu} \tau\).
  \begin{subproblems}
    \item Show that \(a^\mu u_\mu = 0\).
    \item Show that \(a^\mu a_\mu \ge 0\).
    \item Show that
      \[
        a^\mu = \gamma^4 \paren* {
          \frac {\vec v} c \cdot \vec a,
          \vec a + \frac {\vec v} c 
          \times \paren* {\frac {\vec v} c \times \vec a}
        }
      \]
      in the lab frame.
  \end{subproblems}
  \begin{hint}
    \leavevmode
    \begin{subproblems}
    \item Think of \(u^\nu u_\nu = -c^2\);
    \item Consider the rest frame of the particle.
    \item Use the identity
      \(
        \vec w_1 \times (\vec w_2 \times \vec w_3)
        = (\vec w_1 \cdot \vec w_3) \vec w_2 
        - (\vec w_1 \cdot \vec w_2) \vec w_3
      \).
    \end{subproblems}
  \end{hint}
\end{problem}

\begin{solution}
  
\end{solution}

\begin{problem}
  Show that any solution to the equation
  \[
    (y^2 - a^2) F(y) = 0, \quad a > 0
  \]
  can be written as
  \[
    F(y) = f(y) \delta(y^2 - a^2)
  \]
  for some convenient function \(f(y)\).
  \begin{hint}
    Use the identities
    \begin{align*}
      f(x) \delta(x-a) &= f(a) \delta(x-a), \\
      \delta(x^2-a^2) &= \frac 1 {2 \abs a}.
    \end{align*}
  \end{hint}
\end{problem}

\begin{solution}
  
\end{solution}

\begin{problem}
  Consider a spherically symmetric non-static four-current
  \begin{align*}
    j^0(t, \vec x) &= \rho(t, x), \\
    \vec j(t, \vec x) &= \frac {\vec x} x j(t, x)
  \end{align*}
  with \(j^\mu(t, \vec x) = 0\) for \(x > R\) and all \(t\).

  Show that the \(\vec E\) and \(\vec B\) in empty space for \(x>R\) is
  \emph{static} and given by
  \[
    \vec E = Q \frac {\vec x} {x^3}, \qquad \vec B = 0
  \]
  where \(Q = \int \rho(t, x) \d V\).

  \begin{hint}
    The symmetry implies
    \[
      \vec E = \vec x f(t, x), \qquad \vec B = \vec x g(t, x).
    \]
  \end{hint}
\end{problem}

\begin{solution}
  
\end{solution}

\end{document}
