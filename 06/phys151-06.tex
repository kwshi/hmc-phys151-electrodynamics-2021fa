\documentclass{phys151}

\title{Homework 6}
\date{2021 October 28 (Thursday)}
\author{}

\begin{document}

\begin{problem}
  For the problem of a point charge near a planar dielectric considered in
  lecture, compute the volume and surface polarization charges \(\rho_b\) and
  \(\sigma_b\).
\end{problem}

\begin{solution}

\end{solution}

\begin{problem}
  Consider a spherical dielectric at radius \(R\) and dielectric constant
  \(\epsilon\) in vacuum in the presence of a uniform external electric field
  \(E_0\).
  \begin{subproblems}
  \item Using spherical coordinates, write \(\phi\) for \(r<R\) and \(r>R\)
    in terms of an expansion in special functions.
  \item Applying boundary conditions at \(\infty\) (e.g., \(\phi \to -E_0 r
    \cos \theta\)) and at \(r=R\), find \(\phi\) everywhere.
  \item Find the electric field everywhere.
  \item Find the sphere's dipole moment.
  \end{subproblems}
\end{problem}

\begin{solution}

\end{solution}

\begin{problem}
  Find \(\vec E\) everywhere for a spherical cavity of vacuum of radius \(R\)
  inside a dielectric medium with dielectric constant \(\epsilon\) and a
  uniform external electric field \(\vec E_0\).
  \begin{hint}
    See previous problem.
  \end{hint}
\end{problem}

\begin{solution}

\end{solution}

\begin{problem}
  Two long, coaxial cylindrical conductors of radii \(a\) and \(b\) are
  submerged perpendicular into a dielectric liquid with permeability
  \(\chi_e\).  After the cylinders are charged, the liquid rises up and a
  voltage \(V_0\) is measured.  Find the height to which the liquid rises in
  the gap.  Ground the outer conductor and let the density of the dielectric be
  \(\rho\).
\end{problem}

\begin{solution}

\end{solution}

\end{document}
